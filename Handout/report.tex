\documentclass[11pt]{article}
\usepackage{geometry}                
\geometry{letterpaper}                   

\usepackage{graphicx}
\usepackage{amssymb}
\usepackage{epstopdf}
\usepackage{natbib}
\usepackage{amssymb, amsmath}
\usepackage{hyperref}
\usepackage{float}
\usepackage{subcaption}
\DeclareGraphicsRule{.tif}{png}{.png}{`convert #1 `dirname #1`/`basename #1 .tif`.png}
\title{FSK Implementation in Matlab}

\begin{document}


\section*{Introduction}
Unser Ziel bei unserem Projekt war es mithilfe von Matlab Bits FSK-moduliert zu übermitteln über ein Audiosignal. Dafür haben wir als Grundlage die Ideen aus dem P\& S Bits on Air verwendet.
\section*{Implementation}
Um Bits zu übermitteln durchläuft unser Programm folgende Funktionen:

\textbf{Startkeys und Endkeys:} Um Anfang und Ende des Bitstreams zu erkennen wurden jeweils Start- und Endkeys angeh\"angt.

\textbf{Modulation:} Wir haben eine FSK-Modulation benutzt (siehe Handout S. 8) mit $2^n$ Frequenzen, wobei $n$ wählbar ist. 
\end{document}
